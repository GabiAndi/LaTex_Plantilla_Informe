\section{Sección}

Este capítulo es para mostrarle como es la plantilla \cite{EXAMPLE_BOOK}.

\subsection{Subsección}

Puede hacer cajas de colores:

\begin{info}{Título de la caja}
    Y bueno aca el contenido de la caja jajajaj.
\end{info}

\subsubsection{Subsubsección}

También pude definir tipos de cajas para reutilizarlos:

\begin{theo}{Sumatoria de números}{id_para_reconocer}
    Para todo $n$ natural:

    \begin{equation}
        \sum\limits_{i=1}^n i = \frac{n(n+1)}{2}
    \end{equation}
\end{theo}

Luego decimos que en \ref{teorema:id_para_reconocer} esta el secreto del universo.

\section{Lista de tareas}

\begin{todolist}
    \item{Abstract.}
    \item[\done]{Descripción del proyecto:}
    \begin{todolist}
        \item[\done]{Introducción a la problematica.}
        \item[\done]{Análisis de solución.}
        \item[\done]{Planificación del proyecto.}
    \end{todolist} 
    \item[\done]{Desarrollo técnico:}
    \begin{todolist}
        \item[\done]{Selección de componentes.}
        \item[\done]{Diseño del sistema.}
        \item[\done]{Cálculo de presupuesto.}
        \item[\done]{Pruebas en prototipo.} 
    \end{todolist} 
    \item{Conclusiones.}
    \item[\done]{Anexos:}
    \begin{todolist}
        \item[\done]{Esquemas y planos.}
        \item[\done]{Desarrollos complementarios.}
    \end{todolist} 
\end{todolist}